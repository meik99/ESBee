\section{Background}\label{sec:background}

This solution heavily uses active components, which are components that require an electric current.
The main component is a microprocessor, which controls other active parts.
Used technologies are listed and discussed in this section.

\textbf{Microprocessor}: In order to control peripheral components, a processor is needed.
This processor must have a low power consumption and must be programmable.
Furthermore, it is required that some connectivity options are available.
For these reasons, the ESP32\footnote{\url{https://www.espressif.com/en/products/socs/esp32}} processor has been chosen.
It can be programmed using the C++ programming language and has integrated support for Bluetooth and WiFi connections.
Additionally, it is cheap and has a large community, which increases the amount of resources for support and development.

\textbf{C++ and PlatformIO}: C++\footnote{\url{https://en.wikipedia.org/wiki/C\%2B\%2B}} is a well known programming language.
Due to its age and not having a high abstraction from the hardware, it is often regarded as being more complicated to work with than other, more modern languages.
However, there is not yet sufficient support for the ESP32 in other languages.
There are projects such as TinyGo\footnote{\url{https://tinygo.org/}} or Espruino\footnote{\url{https://www.espruino.com/ESP32}}.
They allow running code written in Golang or Javascript respectively on an ESP32, but they do not yet implement the necessary features such as support for bluetooth connections.

PlatformIO\footnote{\url{https://platformio.org/}} is a C++ framework used to ease the development of software for the ESP32.
It integrates a package manager to automatically download and include third party libraries.
Furthermore, it includes an array of tooling to compile code to an executable binary and flash or install it on the microprocessor.

\textbf{Android}: Android\footnote{\url{https://www.android.com/}} is a widely used operating system primarily for smartphones.
Applications for Android are usually written in either the Java or Kotlin programming languages.
In this project, such an application is used to provide a user interface for the microprocessor.

\textbf{DHT11}: A requirement of this project is to measure the outside temperature and react to it.
In order to implement that requirement a temperature sensor is needed.
The DHT11\footnote{\url{https://learn.adafruit.com/dht}} is a temperature and humidity sensor, that is widely used for such projects.

\textbf{Powerbank}: A powerbank is a component that is able to store and release an electric charge.
Since bee hives are typically not in the vicinity of power outlets, a powerbank is needed to provide electrical power to the active components of this system.
They often hold between 5 A and 30 A of charge.

\textbf{Relay}: A relay is an electrical component, that can open or close an electric circuit depending on some electrical input.
An important aspect of a relay is, that the circuit controlling it and the circuit that is being controlled are physically seperate.
This allows controlling a circuit without risking damage to the controlling one.
For example, a circuit that must be limited to 5 V of direct current (DC) to protect components can safely control a circuit that uses 240 V of alternating current (AC).

\textbf{Peristaltic pump}: A peristaltic pump is used to transport the formic acid from a tank to the bee hive.
The advantages of such a pump us that it does not have to be submerged in the fluid to be pumped.
It can also be controlled by supplying or not supplying an electric current with low voltages such as the 5 V typically provided by a powerbank.

\textbf{Formic acid}: Formic acid is a naturally occurring acid, often produced by ants.
The most important characteristic of this acid is its melting point of 8.4°C\cite{FormicAcid}.
This melting point is significant as the goal is to evaporate it into a hive.
Below this temperature, it is not physically possible to do so at normal pressure levels.