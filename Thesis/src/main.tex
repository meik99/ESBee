%! Author = michael
%! Date = 6/20/22

% Preamble
\documentclass[11pt]{article}

% Packages
\usepackage[backend=biber]{biblatex}
\usepackage[a4paper, total={6in, 8in}, margin=2cm]{geometry}
\usepackage{amsmath}
\usepackage{lipsum}

\usepackage[skip=10pt plus1pt, indent=0pt]{parskip}
\usepackage{hyperref}
\usepackage{listings}
\usepackage{xcolor}
\usepackage{graphicx}

\addbibresource{main.bib}

\lstset{language=C++,
    basicstyle=\ttfamily,
    keywordstyle=\color{blue}\ttfamily,
    stringstyle=\color{red}\ttfamily,
    commentstyle=\color{green}\ttfamily,
    morecomment=[l][\color{magenta}]{\#}
}



\title{Autonomously distributing formic acid \linebreak to treat bee hives against varroa mites }
\author{Michael Rynkiewicz, k11736476}

\date{\today}

% Document
\begin{document}
    \maketitle

    \begin{abstract}
        \textbf{Context:} Varroa mites are a threat to bee hives and a major problem bee-keepers face today. \\
        \textbf{Objective:} This paper aims to help bee-keepers treat their bee hives using formic acid. \\
        \textbf{Method:} A system is implemented to distribute this formic acid autonomously in correct dosages, to efficiently kill Varroa mites infesting a bee hive. \\
        \textbf{Results:} The result is a cheap and precise system combining different active components to do achieve this goal. \\
        \textbf{Conclusion:} It is concluded that this system is precise and cheap in absolute, yet more expensive relatively to alternatives. \\
    \end{abstract}

    \tableofcontents
    \newpage

    \input{introduction}

    \section{Background}\label{sec:background}

This solution heavily uses active components, which are components that require an electric current.
The main component is a microprocessor, which controls other active parts.
Used technologies are listed and discussed in this section.

\textbf{Microprocessor}: In order to control peripheral components, a processor is needed.
This processor must have a low power consumption and must be programmable.
Furthermore, it is required that some connectivity options are available.
For these reasons, the ESP32\footnote{\url{https://www.espressif.com/en/products/socs/esp32}} processor has been chosen.
It can be programmed using the C++ programming language and has integrated support for Bluetooth and WiFi connections.
Additionally, it is cheap and has a large community, which increases the amount of resources for support and development.

\textbf{C++ and PlatformIO}: C++\footnote{\url{https://en.wikipedia.org/wiki/C\%2B\%2B}} is a well known programming language.
Due to its age and not having a high abstraction from the hardware, it is often regarded as being more complicated to work with than other, more modern languages.
However, there is not yet sufficient support for the ESP32 in other languages.
There are projects such as TinyGo\footnote{\url{https://tinygo.org/}} or Espruino\footnote{\url{https://www.espruino.com/ESP32}}.
They allow running code written in Golang or Javascript respectively on an ESP32, but they do not yet implement the necessary features such as support for bluetooth connections.

PlatformIO\footnote{\url{https://platformio.org/}} is a C++ framework used to ease the development of software for the ESP32.
It integrates a package manager to automatically download and include third party libraries.
Furthermore, it includes an array of tooling to compile code to an executable binary and flash or install it on the microprocessor.

\textbf{Android}: Android\footnote{\url{https://www.android.com/}} is a widely used operating system primarily for smartphones.
Applications for Android are usually written in either the Java or Kotlin programming languages.
In this project, such an application is used to provide a user interface for the microprocessor.

\textbf{DHT11}: A requirement of this project is to measure the outside temperature and react to it.
In order to implement that requirement a temperature sensor is needed.
The DHT11\footnote{\url{https://learn.adafruit.com/dht}} is a temperature and humidity sensor, that is widely used for such projects.

\textbf{Powerbank}: A powerbank is a component that is able to store and release an electric charge.
Since bee hives are typically not in the vicinity of power outlets, a powerbank is needed to provide electrical power to the active components of this system.
They often hold between 5 A and 30 A of charge.

\textbf{Relay}: A relay is an electrical component, that can open or close an electric circuit depending on some electrical input.
An important aspect of a relay is, that the circuit controlling it and the circuit that is being controlled are physically seperate.
This allows controlling a circuit without risking damage to the controlling one.
For example, a circuit that must be limited to 5 V of direct current (DC) to protect components can safely control a circuit that uses 240 V of alternating current (AC).

\textbf{Peristaltic pump}: A peristaltic pump is used to transport the formic acid from a tank to the bee hive.
The advantages of such a pump us that it does not have to be submerged in the fluid to be pumped.
It can also be controlled by supplying or not supplying an electric current with low voltages such as the 5 V typically provided by a powerbank.

\textbf{Formic acid}: Formic acid is a naturally occurring acid, often produced by ants.
The most important characteristic of this acid is its melting point of 8.4°C\cite{FormicAcid}.
This melting point is significant as the goal is to evaporate it into a hive.
Below this temperature, it is not physically possible to do so at normal pressure levels.

    \section{Requirements and implementation}\label{sec:requirements-and-implementation}

\input{requirements/power consumption.tex}

\subsection{Reading temperature}\label{subsec:reading-temperature}

The next requirement is reading the outside temperature.
This step is necessary to react to the environment and to stop or increase distribution of the formic acid.
As previously established, delivering formic acid below 10 °C must be stopped, as its melting point is already at 8.4 °C.
Furthermore, the following volume per hour of distribution depending on the temperature are established as part of this requirement.
The distribution is also stopped at temperature exceeding 30 °C, as measurement have shown that the humidity is usually too high to effectively evaporate formic acid.

\begin{table}[h]
    \centering
    \caption{Volume per hour depending on outside temperature}
    \label{tab:volume-per-hour-depending-on-temperature}
    \renewcommand{\arraystretch}{1.2}
    \begin{tabular}{l|l}
        Temperature range & Volume per hour [ml/h] \\
        \hline
        $\leq$ 10 °C & 0 \\
        $ > $ 10 °C $\leq$ 15 °C & 0.5 \\
        $ > $ 15 °C $\leq$ 20 °C & 1 \\
        $ > $ 20 °C $\leq$ 25 °C & 1.5 \\
        $ > $ 25 °C $\leq$ 30 °C & 1 \\
        $ > $ 30 °C & 0
    \end{tabular}
\end{table}

\newpage
\begin{lstlisting}[label={lst:react-to-temperature},language=C++, caption=Determining volume based on temperature]
float getPumpAmount(float temperature) {
    if (temperature <= 10.0f) {
        return 0.0f;
    } else if(temperature <= 15.0f) {
        return 0.5f;
    } else if(temperature <= 20.0f) {
        return 1.0f;
    } else if(temperature <= 25.0f) {
        return 1.5f;
    } else if(temperature <= 30.0f) {
        return 1.0f;
    }

    return 0.0f;
}

bool mustContinuePumping(float temperature) {
    return getPump()->pumpedMilliliter(millis()) < getPumpAmount(temperature);
}

void loop() {
     if (millis() - lastSensorReadingsTime >= ONE_HOUR) {
        SensorReadings sensorReadings = readHumidityAndTemperature();
        lastSensorReadings = sensorReadings;
        ...
    }
    ...
    if (millis() - lastPumpCycle >= ONE_HOUR) {
        ...
        getPump()->start(millis());

        while(mustContinuePumping(lastSensorReadings.temperature)) {
            delay(100);
        }
        ...
            getPump()->stop();
        ...
    }
}
\end{lstlisting}

In Listing \ref{lst:react-to-temperature}, the code to determining if the pump should be active or not is shown.
The function \textit{getPumpAmount} returns the amount to be pumped in milliliters, based on the input \textit{temperature}.
This value is used in \textit{mustContinuePumping} to return true, if the pump must still be pumping to reach that value, or false otherwise.

In the main loop of the program, first the temperature is read.
As previously described, this is done once every hour.
The pump is then started and the program is halted as long as \textit{mustContinuePumping} returns true.
Since the pump was started before, it is still running while the program is halted.
After the function returns false, the pump is stopped again.

\begin{figure}[h]
    \centering
    \includegraphics[width=0.5\textwidth]{img/humidity-architecture}
    \caption{Abstraction of the DHT11 library and its usage}
    \label{fig:abstraction-of-dht11}
\end{figure}

\newpage

The temperature itself is read using a library made to interact with the DHT11 sensor.
This interaction is highly abstracted, which makes it unpractical to show the code itself.
In Figure \ref{fig:abstraction-of-dht11}, this abstraction is visualised.
The main loop only uses a high level abstraction, which provides a function to retrieve the temperature.
A class that interacts with the library, i.e., the sensor, implements this abstraction.
This architecture allows switching sensor types, if necessary, with only minimal code changes.


\subsection{Activating the fan}\label{subsec:activating-the-fan}


    \section{Summary and conclusion}\label{sec:conclusion}

In this paper, the implementation of a system to treat bee hives with formic acid has been shown.
At the beginning, the reasoning behind this project, the threat of Varroa mites to bees, was established.
Different approaches to this problem have been discussed and background technology has been explained.
Finally, the requirements of the system have been listed and their implementation presented.

In conclusion, a power and cost-efficient system has been created, that has the potential to help bee-keepers fight Varroa mites.
Advantages of the system include the manual control of an application and the precise timing and control of a processor.
Additionally, the system can be expanded to include an internet connection to remotely control it in the future.
Disadvantages are the reliance on electric power and that it is more expensive than alternatives.
It is cost-effective in an absolute sense, relative to some alternatives, however, it can be more expensive.


    \newpage
    \printbibliography

\end{document}
