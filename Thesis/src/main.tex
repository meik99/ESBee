%! Author = michael
%! Date = 6/20/22

% Preamble
\documentclass[11pt]{article}

% Packages
\usepackage[backend=biber]{biblatex}
\usepackage[a4paper, total={6in, 8in}, margin=2cm]{geometry}
\usepackage{amsmath}
\usepackage{lipsum}

\usepackage[skip=10pt plus1pt, indent=0pt]{parskip}
\usepackage{hyperref}
\usepackage{listings}
\usepackage{xcolor}
\usepackage{graphicx}

\addbibresource{main.bib}

\lstset{language=C++,
    basicstyle=\ttfamily,
    keywordstyle=\color{blue}\ttfamily,
    stringstyle=\color{red}\ttfamily,
    commentstyle=\color{green}\ttfamily,
    morecomment=[l][\color{magenta}]{\#}
}



\title{Autonomously distributing formic acid \linebreak to treat bee hives against varroa mites }
\author{Michael Rynkiewicz, k11736476}

\date{\today}

% Document
\begin{document}
    \maketitle

    \begin{abstract}
        \textbf{Context:} Varroa mites are a threat to bee hives and a major problem bee-keepers face today. \\
        \textbf{Objective:} This paper aims to help bee-keepers treat their bee hives using formic acid. \\
        \textbf{Method:} A system is implemented to distribute this formic acid autonomously in correct dosages, to efficiently kill Varroa mites infesting a bee hive. \\
        \textbf{Results:} The result is a cheap and precise system combining different active components to do achieve this goal. \\
        \textbf{Conclusion:} It is concluded that this system is precise and cheap in absolute, yet more expensive relatively to alternatives. \\
    \end{abstract}

    \tableofcontents
    \newpage

    \section{Introduction}\label{sec:introduction}

In 2006, beekeepers started to observe a sharp decline in honey bee colonies\cite{ColonyCollapseDisorder}.
The estimated loss of beehives between 2007 and 2008 was 35.8 \%.
This decline has been called colony collapse disorder and is a problem to this day.

Current research suggests that not one specific event is causing this disorder, but a collection of issues\cite{VarroaMitesAndHoneyBeeHealth}.
One of which is the Varroa mite, as suggested by Le Conte et.al.\cite{VarroaMitesAndHoneyBeeHealth}.
Guzmán-Novoa et.al.\ suggest that the Varroa mite is the largest contributor to colony mortality, associated with 85 \% of colony deaths\cite{VarroaDestructorIsTheMainCulprit}.
Furthermore, Wilfert et.al.\ have shown, that the Varroa mite is a vector for the deformed wing virus, further increasing pressure on colony health\cite{DeformedWingVirusDueToVarroa}.

Combating the Varroa mite is challenging task.
Pesticides may taint the produced honey, decreasing its quality, or put more pressure on the colony health\cite{PesticidesInHoney}.
In order to circumvent these issues, organic alternatives are being researched.
Gregorc and Sampson identified various effective options, one of which is the application of formic acid\cite{DiagnosisOfVarroaMiteAndSustainableControl}.
They determined that, depending on the concentration and time of application, formic acid kills between 43 \% and 97 \% of Varroa mites in a bee hive.
It's high volatility lets it evaporate quickly, decreasing the probability of longer lasting residue effecting the hive or the honey.

Gregorc and Sampson, however, identified a problem with formic acid\cite{DiagnosisOfVarroaMiteAndSustainableControl}.
At higher temperatures, bees cover vents of their hive to regulate the temperature within.
Meaning less formic acid can make contact with the mites, decreasing mite mortality and increasing brood mortality.

Various ways of applying formic acid exist today, with varying degrees of effectiveness.
Gel packets, distributing formic acid over a two week period, showed mite mortalities of 93.6 \% to 100 \%\cite{FormicAcidBasedTreatments}.
Using a so call Liebig-Dispenser, a paper wick soaked in formic acid, showed similar results\cite{FormicAcidBasedTreatments}.
Another cheap and effective solution is a formic acid fumigator\cite{FormicAcidFumigator}.
A passive fumigator hood, applied to a bee hive, that uses the ventilation produced by bees to fumigate the hive, showed an estimate of an 80 \% reduction of Varroa mite population.

This paper describes an additional way of applying formic acid to a beehive.
Using active, i.e., powered, components, a bee hive can be fumigated with formic acid.
Since active components are used, more control and the possibility of observing a bee hive over distance is attainable.

First, some technical information is established.
Technologies and components that are used are described.
For each of these technologies, a short general description exists and the reason a particular technology or component is used is added.
Then, the requirements and their implementations are discussed.
In that section, a specific requirement of the system is first described, then the implementation is shown and discussed.
Finally, a conclusion is summarizing and closing this paper.


    \section{Background}\label{sec:background}

This solution heavily uses active components, which are components that require an electric current.
The main component is a microprocessor, which controls other active parts.
Used technologies are listed and discussed in this section.

\textbf{Microprocessor}: In order to control peripheral components, a processor is needed.
This processor must have a low power consumption and must be programmable.
Furthermore, it is required that some connectivity options are available.
For these reasons, the ESP32\footnote{\url{https://www.espressif.com/en/products/socs/esp32}} processor has been chosen.
It can be programmed using the C++ programming language and has integrated support for Bluetooth and WiFi connections.
Additionally, it is cheap and has a large community, which increases the amount of resources for support and development.

\textbf{C++ and PlatformIO}: C++\footnote{\url{https://en.wikipedia.org/wiki/C\%2B\%2B}} is a well known programming language.
Due to its age and not having a high abstraction from the hardware, it is often regarded as being more complicated to work with than other, more modern languages.
However, there is not yet sufficient support for the ESP32 in other languages.
There are projects such as TinyGo\footnote{\url{https://tinygo.org/}} or Espruino\footnote{\url{https://www.espruino.com/ESP32}}.
They allow running code written in Golang or Javascript respectively on an ESP32, but they do not yet implement the necessary features such as support for bluetooth connections.

PlatformIO\footnote{\url{https://platformio.org/}} is a C++ framework used to ease the development of software for the ESP32.
It integrates a package manager to automatically download and include third party libraries.
Furthermore, it includes an array of tooling to compile code to an executable binary and flash or install it on the microprocessor.

\textbf{Android}: Android\footnote{\url{https://www.android.com/}} is a widely used operating system primarily for smartphones.
Applications for Android are usually written in either the Java or Kotlin programming languages.
In this project, such an application is used to provide a user interface for the microprocessor.

\textbf{DHT11}: A requirement of this project is to measure the outside temperature and react to it.
In order to implement that requirement a temperature sensor is needed.
The DHT11\footnote{\url{https://learn.adafruit.com/dht}} is a temperature and humidity sensor, that is widely used for such projects.

\textbf{Powerbank}: A powerbank is a component that is able to store and release an electric charge.
Since bee hives are typically not in the vicinity of power outlets, a powerbank is needed to provide electrical power to the active components of this system.
They often hold between 5 A and 30 A of charge.

\textbf{Relay}: A relay is an electrical component, that can open or close an electric circuit depending on some electrical input.
An important aspect of a relay is, that the circuit controlling it and the circuit that is being controlled are physically seperate.
This allows controlling a circuit without risking damage to the controlling one.
For example, a circuit that must be limited to 5 V of direct current (DC) to protect components can safely control a circuit that uses 240 V of alternating current (AC).

\textbf{Peristaltic pump}: A peristaltic pump is used to transport the formic acid from a tank to the bee hive.
The advantages of such a pump us that it does not have to be submerged in the fluid to be pumped.
It can also be controlled by supplying or not supplying an electric current with low voltages such as the 5 V typically provided by a powerbank.

\textbf{Formic acid}: Formic acid is a naturally occurring acid, often produced by ants.
The most important characteristic of this acid is its melting point of 8.4°C\cite{FormicAcid}.
This melting point is significant as the goal is to evaporate it into a hive.
Below this temperature, it is not physically possible to do so at normal pressure levels.

    \section{Requirements and implementation}\label{sec:requirements-and-implementation}

\subsection{Power consumption}\label{subsec:power-consumption}



    \section{Summary and conclusion}\label{sec:conclusion}

In this paper, the implementation of a system to treat bee hives with formic acid has been shown.
At the beginning, the reasoning behind this project, the threat of Varroa mites to bees, was established.
Different approaches to this problem have been discussed and background technology has been explained.
Finally, the requirements of the system have been listed and their implementation presented.

In conclusion, a power and cost-efficient system has been created, that has the potential to help bee-keepers fight Varroa mites.
Advantages of the system include the manual control of an application and the precise timing and control of a processor.
Additionally, the system can be expanded to include an internet connection to remotely control it in the future.
Disadvantages are the reliance on electric power and that it is more expensive than alternatives.
It is cost-effective in an absolute sense, relative to some alternatives, however, it can be more expensive.


    \newpage
    \printbibliography

\end{document}
