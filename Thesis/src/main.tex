%! Author = michael
%! Date = 6/20/22

% Preamble
\documentclass[11pt]{article}

% Packages
\usepackage[backend=biber]{biblatex}
\usepackage[a4paper, total={6in, 8in}, margin=2cm]{geometry}
\usepackage{amsmath}
\usepackage{lipsum}

\addbibresource{main.bib}

\usepackage[skip=10pt plus1pt, indent=0pt]{parskip}

\title{Autonomously distributing formic acid \linebreak to treat bee hives against varroa mites }
\author{Michael Rynkiewicz, k11736476}

\date{\today}

% Document
\begin{document}
    \maketitle

%    \begin{abstract}
%    \end{abstract}

    \tableofcontents
    \newpage


    \section{Introduction}\label{sec:introduction}

    In 2006, beekeepers started to observe a sharp decline in honey bee colonies\cite{ColonyCollapseDisorder}.
    The estimated loss of beehives between 2007 and 2008 was 35.8 \%.
    This decline has been called colony collapse disorder and is a problem to this day.

    Current research suggests that not one specific event is causing this disorder, but a collection of issues\cite{VarroaMitesAndHoneyBeeHealth}.
    One of which is the Varroa mite, as suggested by Le Conte et.al.\cite{VarroaMitesAndHoneyBeeHealth}.
    Guzmán-Novoa et.al\. suggest that the Varroa mite is the largest contributor to colony mortality, associated with 85 \% of colony deaths\cite{VarroaDestructorIsTheMainCulprit}.
    Furthermore, Wilfert et.al\. have shown, that the Varroa mite is a vector for the deformed wing virus, further increasing pressure on colony health\cite{DeformedWingVirusDueToVarroa}.

    Combating the Varroa mite is challenging task.
    Pesticides may taint the produced honey, decreasing its quality, or put more pressure on the colony health\cite{PesticidesInHoney}.
    In order to circumvent these issues, organic alternatives are being researched.
    Gregorc and Sampson identified various effective options, one of which is the application of formic acid\cite{DiagnosisOfVarroaMiteAndSustainableControl}.
    They determined that, based on the concentration and time of application of formic acid, kills between 43 \% and 97 \% of Varroa mites in a bee hive.
    It's high volatility lets it evaporate quickly, decreasing the probability of longer lasting residue effecting the hive or the honey.

    Gregorc and Sampson, however, identified a problem with formic acid\cite{DiagnosisOfVarroaMiteAndSustainableControl}.
    At higher temperatures, bees cover vents of their hive to regulate the temperature within.
    Meaning less formic acid can make contact with the mites, decreasing mite mortality and increasing brood mortality.

    Various ways of applying formic acid exist today, with varying degrees of effectiveness.
    Gel packets, distributing formic acid over a two week period, showed mite mortalities of 93.6 \% to 100 \%\cite{FormicAcidBasedTreatments}.
    Using a so call Liebig-Dispenser, a paper wick soaked in formic acid, showed similar results\cite{FormicAcidBasedTreatments}.
    Another cheap and effective solution is a formic acid fumigator\cite{FormicAcidFumigator}.
    A passive fumigator hood, applied to a bee hive, that uses the ventilation produced by bees to fumigate the hive, showed an estimate of an 80 \% reduction of Varroa mite population.

    This paper describes an additional way of applying formic acid to a beehive.
    Using active, i.e., powered, components, a bee hive can be fumigated with formic acid.
    Since active components are used, more control and the possibility of observing a bee hive over distance can be increased.

    \section{Background}\label{sec:background}

    \newpage
    \printbibliography

\end{document}
