\section{Requirements and implementation}\label{sec:requirements-and-implementation}

\subsection{Power consumption}\label{subsec:power-consumption}

One of the main requirements of this system is a low power consumption.
Bee hives often remain in remote locations, such as woods or pastures.
Therefore, a power outlet is not an option to power active components.
Furthermore, ample power must be supplied through the whole duration of the treatment, which is about two weeks.

\begin{table}
    \centering
    \caption{Power consumption of active components}
    \label{tab:power-consumption-of-active-components}
    \renewcommand{\arraystretch}{1.2}
    \begin{tabular}{l|l|l|l}
        Amount used & Component & Idle power consumption [Ah] & Active power consumption [Ah] \\
        \hline
        1 & ESP32 &  $0.8 * 10^{-3}$ & $68 * 10^{-3}$ \\
        2 & 5 V relay & 0 & $70 * 10^{-3}$ \\
        1 & 6 V peristaltic pump & 0 & $300 * 10^{-3}$ \\
        1 & 12 V fan & 0 & $200 * 10^{-3} $ \\
        1 & DHT11 & $150 * 10^{-6}$ & $500 * 10^{-3}$ \\
        2 & 130mm x 150mm solar panel & N.A. & $- 500 * 10^{-3}$
    \end{tabular}
\end{table}


In order to fulfill this requirement, the power usage of the active components must be considered.
In Table \ref{tab:power-consumption-of-active-components}, the power consumption of all used components is listed.
Note well that solar panels have also been used.
Their power consumption is listed as a negative value, as they produce power instead of consuming it.
Furthermore, the maximal output voltage is limited to 5 V due to the output of the powerbank or processor.
Components that need a higher voltage also work with a lower voltage, albeit with a reduced performance.

As for the matter of power consumption, in a later section the algorithm used to deliver the formic acid and distribute it is shown in detail. %TODO: reference section as soon as it exists
In order to analyse the theoretical power consumption, a short outline is given here.
Every hour, the DHT11 is used to measure the environments temperature and humidity.
This process takes 30\ s at the highest.
Then, the pump is powered until, at most, 1.5 ml of formic acid has been delivered.
The pumps flow rate is 23 ml/min.
Therefore, the pump must be powered for $\frac{1.5 ml}{23 \frac{ml}{min}} = 0.065\ min = 3.9\ s \approx 4\ s$.
At the same time, the fan is powered for one minute to distribute the formic acid into the bee hive.
Since the microprocessor offers a UI via a bluetooth connection, it is assumed that it runs active for the whole duration.
The UI is discussed in a later section. %TODO: reference section as soon as it exists

\begin{math}
    30\ s = 0.0083\ h \\
    4\ s = 0.0011\ h \\
    1\ m = 0.0167\ h \\
    Q_{dht11} = 500 * 10^{-3}\ A * 0.0083\ h + 150 * 10^{-6}\ A * (1 - 0.0083)\ h = 0.0043\ Ah\\
\end{math}