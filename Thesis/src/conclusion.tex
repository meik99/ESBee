\section{Summary and conclusion}\label{sec:conclusion}

In this paper, the implementation of a system to treat bee hives with formic acid has been shown.
At the beginning, the reasoning behind this project, the threat of Varroa mites to bees, was established.
Different approaches to this problem have been discussed and background technology has been explained.
Finally, the requirements of the system have been listed and their implementation presented.

In conclusion, a power and cost-efficient system has been created, that has the potential to help bee-keepers fight Varroa mites.
Advantages of the system include the manual control of an application and the precise timing and control of a processor.
Additionally, the system can be expanded to include an internet connection to remotely control it in the future.
Disadvantages are the reliance on electric power and that it is more expensive than alternatives.
It is cost-effective in an absolute sense, relative to some alternatives, however, it can be more expensive.
